\documentclass[a4paper,12pt]{article} 

\usepackage[top = 2.5cm, bottom = 2.5cm, left = 2.5cm, right = 2.5cm]{geometry} 

% packages
\usepackage{amsmath, amsfonts, amsthm} % basic math packages
\usepackage{tikz} % for making illustrations
\usetikzlibrary{shapes.arrows, arrows, decorations.markings, positioning}
\usetikzlibrary{calc}
\usetikzlibrary{3d}
\usepackage{graphicx} % for importing images
\usepackage{xcolor} % more color options
\usepackage{colortbl}
\usepackage{multicol} % for making two-column lists
\usepackage{hyperref} % for hyperlinking
%\hypersetup{colorlinks=true, urlcolor=cyan,}
\usepackage{mathabx}
\usepackage{cleveref}
\usepackage{subfig}
\usepackage{array}
\usepackage{wrapfig}
\usepackage{bbm}
\usepackage{fancyhdr}
\usepackage{algorithm, algorithmicx, algpseudocode}
\usepackage{stmaryrd}
\usepackage{physics}


% The following two packages - multirow and booktabs - are needed to create nice looking tables.
\usepackage{multirow} % Multirow is for tables with multiple rows within one cell.
\usepackage{booktabs} % For even nicer tables.

% As we usually want to include some plots (.pdf files) we need a package for that.
\usepackage{graphicx} 

% The default setting of LaTeX is to indent new paragraphs. This is useful for articles. But not really nice for homework problem sets. The following command sets the indent to 0.
\usepackage{setspace}
\setlength{\parindent}{0in}

% Package to place figures where you want them.
\usepackage{float}

% The fancyhdr package let's us create nice headers.
\usepackage{fancyhdr}

% theorems, lemmas, examples, etc.
\newtheorem{theorem}{Theorem}[section]
% \newtheorem{corollary}{Corollary}[theorem]
% \newtheorem{lemma}[theorem]{Lemma}
\newtheorem{example}[theorem]{Example}
\newtheorem{lemma}[theorem]{Lemma}
\theoremstyle{definition}
\newtheorem{definition}{Definition}[section]
\theoremstyle{remark}
\newtheorem*{remark}{Remark}
\newtheorem*{solution}{Solution}

\def\mydefb#1{\expandafter\def\csname bf#1\endcsname{\mathbf{#1}}}
\def\mydefallb#1{\ifx#1\mydefallb\else\mydefb#1\expandafter\mydefallb\fi}
\mydefallb aAbBcCdDeEfFgGhHiIjJkKlLmMnNoOpPqQrRsStTuUvVwWxXyYzZ\mydefallb

\def\mydefb#1{\expandafter\def\csname cal#1\endcsname{\mathcal{#1}}}
\def\mydefallb#1{\ifx#1\mydefallb\else\mydefb#1\expandafter\mydefallb\fi}
\mydefallb aAbBcCdDeEfFgGhHiIjJkKlLmMnNoOpPqQrRsStTuUvVwWxXyYzZ\mydefallb

%% Change this to just the normal N,Z,R,C,P,E
\def\mydefb#1{\expandafter\def\csname bb#1\endcsname{\mathbb{#1}}}
\def\mydefallb#1{\ifx#1\mydefallb\else\mydefb#1\expandafter\mydefallb\fi}
\mydefallb CEGIKNPQRSTZ\mydefallb

\newcommand{\half}{\frac{1}{2}}
\DeclareMathOperator{\sgn}{sgn}
\DeclareMathOperator*{\argmax}{arg\,max}
\DeclareMathOperator*{\argmin}{arg\,min}
\newcommand{\matlab}{\textsc{Matlab}}
\newcommand{\rdist}{\text{r-dist}}
\newcommand{\kdist}{\text{k-dist}}
\newcommand{\dist}{\text{dist}}
\newcommand{\lrd}{\text{lrd}}
\newcommand{\lof}{\text{LOF}}


%%%%%%%%%%%%%%%%%%%%%%%%%%%%%%%%%%%%%%%%%%%%%%%%
% 3. Header (and Footer)
%%%%%%%%%%%%%%%%%%%%%%%%%%%%%%%%%%%%%%%%%%%%%%%%

% To make our document nice we want a header and number the pages in the footer.

\pagestyle{fancy} % With this command we can customize the header style.

\fancyhf{} % This makes sure we do not have other information in our header or footer.

\lhead{\footnotesize CS 584:  Homework  \# 6}% \lhead puts text in the top left corner. \footnotesize sets our font to a smaller size.

%\rhead works just like \lhead (you can also use \chead)
\rhead{\footnotesize Scott (mtscot4)} %<---- Fill in your lastnames.

% Similar commands work for the footer (\lfoot, \cfoot and \rfoot).
% We want to put our page number in the center.
\cfoot{\footnotesize \thepage} 

\begin{document}
	
	
	%%%%%%%%%%%%%%%%%%%%%%%%%%%%%%%%%%%%%%%%%%%%%%%%
	%%%%%%%%%%%%%%%%%%%%%%%%%%%%%%%%%%%%%%%%%%%%%%%%
	
	%%%%%%%%%%%%%%%%%%%%%%%%%%%%%%%%%%%%%%%%%%%%%%%%
	% Title section of the document
	%%%%%%%%%%%%%%%%%%%%%%%%%%%%%%%%%%%%%%%%%%%%%%%%
	
	% For the title section we want to reproduce the title section of the Problem Set and add your names.
	
	\thispagestyle{empty} % This command disables the header on the first page. 
	
	\begin{tabular}{p{15.5cm}} % This is a simple tabular environment to align your text nicely 
		{\large \sc CS 584:  Spatial Algorithms} \\
		Emory University \\ Fall 2024 \\ Prof. Andreas Z\"ufle\\
		\hline % \hline produces horizontal lines.
		\\
	\end{tabular} % Our tabular environment ends here.
	
	\vspace*{0.3cm} % Now we want to add some vertical space in between the line and our title.
	
	\begin{center} % Everything within the center environment is centered.
		{\Large \bf Homework \# 6} % <---- Don't forget to put in the right number
		\vspace{2mm}
		
		% YOUR NAMES GO HERE
		{\bf Mitchell Scott}\\ (mtscot4) % <---- Fill in your names here!
		
	\end{center}  
	
	\vspace{0.4cm}
	
	%%%%%%%%%%%%%%%%%%%%%%%%%%%%%%%%%%%%%%%%%%%%%%%%
	%%%%%%%%%%%%%%%%%%%%%%%%%%%%%%%%%%%%%%%%%%%%%%%%
	
	% Up until this point you only have to make minor changes for every week (Number of the homework). Your write up essentially starts here.
	
	\begin{figure}[h!]
		\centering
		\includegraphics[width=0.7\linewidth]{"GraphForProblem"}
		\caption{}
		\label{fig:graphforproblem}
	\end{figure}
	
\section*{General Formulae}
First recall that there are three main formulas for computing the local outlier factor (LOF):
\begin{align}
	\rdist_k(p,q) &:= \max(\kdist(q), \dist(p,q))\label{eqn:rdist}\\
	\lrd_k(o) &:=\left(\frac{\sum_{o\in N_k(o)}\rdist_k(o,o')}{\norm{N_k(o)}}\right)^{-1}\label{eqn:lrd}\\
	\lof(o) &= \frac{\sum_{o'\in N_k(o)}\frac{\lrd_k(o')}{\lrd_k(o)}}{\norm{N_k(0)}}\label{eqn:lof}
\end{align}

\section*{Applied To Point $L$ with $k=3$}
Since our problem is concerned with $k=3$ looking at point $L$, we can rewrite Eqn.\ref{eqn:lof} in the following way:

\begin{align*}
	\lof_3(L) &:=\frac{\sum_{o'\in N_3(L)}\frac{\lrd_3(o')}{\lrd_3(L)}}{\norm{N_3(L)}}
\end{align*}
First, we need to ascertain what $N_3(L)$ is. Intuitively, $N_k(O)$ is the set of all points that are $k$-nearest neighbors with point $O$. However, this set may not be just $k$ points, especially if there is a tie then we might have to include more points. But for this case, $N_3(L) = \{K, M, O\}$. Since there is only one points 3 away and exactly two points 4 away from $L$, then $\norm{N_3(L)} = 3$. Combining all of this information, we arrive at: 
\begin{align*}
		\lof_3(L) &= \frac{\frac{\lrd_3(K)}{\lrd_3(L)} + \frac{\lrd_3(M)}{\lrd_3(L)} + \frac{\lrd_3(O)}{\lrd_3(L)}}{3}\label{eqn:lor_mod}
\end{align*}

\subsection*{Computing $\lrd_3(K)$}
The first part of Eqn. \ref{eqn:lor_mod} is computing  $lrb_3(K)$.  First, we observe that $N_3(K) = \{J,H,I,G,L\}$, as there is a 3-way tie for the third nearest neighbor, so we have to enclude G,I, and L, which means that $\norm{N_3(K)} = 5$.  Now using eqn. \ref{eqn:lrd}, modified for our problem, we have
\begin{align*}
	\lrd_3(K) &= \left(\frac{\rdist_3(K,J) + \rdist_3(K,H) + \rdist_3(K,I) + \rdist_3(K,G) + \rdist_3(K,L)}{5}\right)^{-1}\\
	&=\left(\max(2,2) + \max(2,3) + \max(3,4) + \max(2,4) + \max(4,4)\right)\\
	&= \left(\frac{2 + 3 + 4 +4+4}{5}\right)^{-1}\\
	&= \left(\frac{17}{5}\right)^{-1}\\
	&=\frac{5}{17}
\end{align*}

\subsection*{Computing $\lrd_3(M)$}
Next, observe that $N_3(M) = \{O,P,R, N\}$. This means that $\norm{N_3(M)} = 4$. Also
\begin{align*}
	\lrd_3(M) &= \left(\frac{\rdist_3(M,O) + \rdist_3(M,P) + \rdist_3(M,R) + \rdist_3(M,N)}{4}\right)^{-1}\\
	&=\left(\frac{\max(1,1) + \max(2,2) + \max(1,2) + \max(2,2)}{4}\right)^{-1}\\
	&= \left(\frac{1 + 2 + 2 +2}{4}\right)^{-1}\\
	&= \left(\frac{7}{4}\right)^{-1}\\
	&=\frac{4}{7}
\end{align*}


\subsection*{Computing $\lrd_3(O)$}
Now observe that $N_3(O) = \{M,N,R,P\}$. This means that $\norm{N_3(O)} = 4$. Also,
\begin{align*}
	\lrd_3(O) &= \left(\frac{\rdist_3(O,M) + \rdist_3(O,N) + \rdist_3(O,R) + \rdist_3(O,P)}{4}\right)^{-1}\\
	&=\left(\frac{\max(2,1) + \max(2,2) + \max(1,1) + \max(2,1)}{4}\right)^{-1}\\
	&= \left(\frac{2 + 2 + 1 +2}{4}\right)^{-1}\\
	&= \left(\frac{7}{4}\right)^{-1}\\
	&=\frac{4}{7}
\end{align*}


\subsection*{Computing $\lrd_3(L)$}
Now observe that $N_3(L) = \{M,O,K\}$. This means that $\norm{N_3(L)} = 3$. Also,
\begin{align*}
	\lrd_3(L) &= \left(\frac{\rdist_3(L,M) + \rdist_3(L,O) + \rdist_3(L,K)}{3}\right)^{-1}\\
	&=\left(\frac{\max(3,2) + \max(1,4) + \max(3,4)}{3}\right)^{-1}\\
	&= \left(\frac{3 + 4 + 4}{3}\right)^{-1}\\
	&= \left(\frac{11}{3}\right)^{-1}\\
	&=\frac{3}{11}
\end{align*}

\subsection*{Final Computation}
Now that we have computed all of the preliminary steps, we can compute the local outlier factor, in eqn \ref{eqn:lor_mod}.  We see that
\begin{align*}
	\lof_3(L) &= \frac{\frac{\lrd_3(K)}{\lrd_3(L)} + \frac{\lrd_3(M)}{\lrd_3(L)} + \frac{\lrd_3(O)}{\lrd_3(L)}}{3}\\
	&= \frac{\frac{\frac{5}{17}}{\frac{3}{11}}+ \frac{\frac{5}{17}}{\frac{3}{11}} + \frac{\frac{5}{17}}{\frac{3}{11}}}{3}\\
	&= \frac{\frac{55}{51} + \frac{44}{21} + \frac{44}{21}}{3}\\
	&= \frac{385}{357} + \frac{748}{357} + \frac{748}{357}\\
	&= \frac{1881}{357\cdot 3}\\
	&= \frac{627}{357} \approx 1.75630252
\end{align*} 



	
\section*{Acknowledgements}
	I would like to acknowledge that I worked with fellow CS 584 students 
	
\end{document}
