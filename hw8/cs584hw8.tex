\documentclass[a4paper,12pt]{article} 

\usepackage[top = 2.5cm, bottom = 2.5cm, left = 2.5cm, right = 2.5cm]{geometry} 

% packages
\usepackage{amsmath, amsfonts, amsthm} % basic math packages
\usepackage{tikz} % for making illustrations
\usetikzlibrary{shapes.arrows, arrows, decorations.markings, positioning}
\usetikzlibrary{calc}
\usetikzlibrary{3d}
\usepackage{graphicx} % for importing images
\usepackage{xcolor} % more color options
\usepackage{colortbl}
\usepackage{multicol} % for making two-column lists
\usepackage{hyperref} % for hyperlinking
%\hypersetup{colorlinks=true, urlcolor=cyan,}
\usepackage{mathabx}
\usepackage{cleveref}
\usepackage{subfig}
\usepackage{array}
\usepackage{wrapfig}
\usepackage{bbm}
\usepackage{fancyhdr}
\usepackage{algorithm, algorithmicx, algpseudocode}
\usepackage{stmaryrd}
\usepackage{physics}


% The following two packages - multirow and booktabs - are needed to create nice looking tables.
\usepackage{multirow} % Multirow is for tables with multiple rows within one cell.
\usepackage{booktabs} % For even nicer tables.

% As we usually want to include some plots (.pdf files) we need a package for that.
\usepackage{graphicx} 

% The default setting of LaTeX is to indent new paragraphs. This is useful for articles. But not really nice for homework problem sets. The following command sets the indent to 0.
\usepackage{setspace}
\setlength{\parindent}{0in}

% Package to place figures where you want them.
\usepackage{float}

% The fancyhdr package let's us create nice headers.
\usepackage{fancyhdr}

% theorems, lemmas, examples, etc.
\newtheorem{theorem}{Theorem}[section]
% \newtheorem{corollary}{Corollary}[theorem]
% \newtheorem{lemma}[theorem]{Lemma}
\newtheorem{example}[theorem]{Example}
\newtheorem{lemma}[theorem]{Lemma}
\theoremstyle{definition}
\newtheorem{definition}{Definition}[section]
\theoremstyle{remark}
\newtheorem*{remark}{Remark}
\newtheorem*{solution}{Solution}

\def\mydefb#1{\expandafter\def\csname bf#1\endcsname{\mathbf{#1}}}
\def\mydefallb#1{\ifx#1\mydefallb\else\mydefb#1\expandafter\mydefallb\fi}
\mydefallb aAbBcCdDeEfFgGhHiIjJkKlLmMnNoOpPqQrRsStTuUvVwWxXyYzZ\mydefallb

\def\mydefb#1{\expandafter\def\csname cal#1\endcsname{\mathcal{#1}}}
\def\mydefallb#1{\ifx#1\mydefallb\else\mydefb#1\expandafter\mydefallb\fi}
\mydefallb aAbBcCdDeEfFgGhHiIjJkKlLmMnNoOpPqQrRsStTuUvVwWxXyYzZ\mydefallb

%% Change this to just the normal N,Z,R,C,P,E
\def\mydefb#1{\expandafter\def\csname bb#1\endcsname{\mathbb{#1}}}
\def\mydefallb#1{\ifx#1\mydefallb\else\mydefb#1\expandafter\mydefallb\fi}
\mydefallb CEGIKNPQRST\mydefallb

\newcommand{\half}{\frac{1}{2}}
\DeclareMathOperator{\sgn}{sgn}
\DeclareMathOperator*{\argmax}{arg\,max}
\DeclareMathOperator*{\argmin}{arg\,min}
\newcommand{\matlab}{\textsc{Matlab}}

\newcommand{\g}{\nabla f(\bfx_0)}

%%%%%%%%%%%%%%%%%%%%%%%%%%%%%%%%%%%%%%%%%%%%%%%%
% 3. Header (and Footer)
%%%%%%%%%%%%%%%%%%%%%%%%%%%%%%%%%%%%%%%%%%%%%%%%

% To make our document nice we want a header and number the pages in the footer.

\pagestyle{fancy} % With this command we can customize the header style.

\fancyhf{} % This makes sure we do not have other information in our header or footer.

\lhead{\footnotesize CS 584: Homework  \# 8}% \lhead puts text in the top left corner. \footnotesize sets our font to a smaller size.

%\rhead works just like \lhead (you can also use \chead)
\rhead{\footnotesize Scott (mtscot4)} %<---- Fill in your lastnames.

% Similar commands work for the footer (\lfoot, \cfoot and \rfoot).
% We want to put our page number in the center.
\cfoot{\footnotesize \thepage} 

\begin{document}
	
	
	%%%%%%%%%%%%%%%%%%%%%%%%%%%%%%%%%%%%%%%%%%%%%%%%
	%%%%%%%%%%%%%%%%%%%%%%%%%%%%%%%%%%%%%%%%%%%%%%%%
	
	%%%%%%%%%%%%%%%%%%%%%%%%%%%%%%%%%%%%%%%%%%%%%%%%
	% Title section of the document
	%%%%%%%%%%%%%%%%%%%%%%%%%%%%%%%%%%%%%%%%%%%%%%%%
	
	% For the title section we want to reproduce the title section of the Problem Set and add your names.
	
	\thispagestyle{empty} % This command disables the header on the first page. 
	
	\begin{tabular}{p{15.5cm}} % This is a simple tabular environment to align your text nicely 
		{\large \sc CS 584: Spatial Algorithms} \\
		Emory University \\ Fall 2024 \\ Prof. Andreas Z\"ufle \\
		\hline % \hline produces horizontal lines.
		\\
	\end{tabular} % Our tabular environment ends here.
	
	\vspace*{0.3cm} % Now we want to add some vertical space in between the line and our title.
	
	\begin{center} % Everything within the center environment is centered.
		{\Large \bf Homework \# 8} % <---- Don't forget to put in the right number
		\vspace{2mm}
		
		% YOUR NAMES GO HERE
		{\bf Mitchell Scott}\\ (mtscot4) % <---- Fill in your names here!
		
	\end{center}  
	
	\vspace{0.4cm}
	
	%%%%%%%%%%%%%%%%%%%%%%%%%%%%%%%%%%%%%%%%%%%%%%%%
	%%%%%%%%%%%%%%%%%%%%%%%%%%%%%%%%%%%%%%%%%%%%%%%%
	
	% Up until this point you only have to make minor changes for every week (Number of the homework). Your write up essentially starts here.
	
	
	\begin{enumerate}
		
		\item {\bf Compute the probability that at least three wolves are close to the group! You may use the Generating Function based approach covered in class}. 
			\begin{solution}
				We start with the Generating Function based method: For the wolves $w_i, i\in\{1,\cdots,5\}$, they have $\bbP(w_i) = \{0.8,0.5,0.2,0.1,0.3\}$ of being ``close to the group of tourists" . This means $1-\bbP(w_i) = \{0.2,0.5,0.8,0.9,0.7\}$ of the wolves being ``far from the group of tourists".  Then the Generating Function Method becomes:
				\begin{align*}
					\prod_{i=1}^5 \bbP(w_i) w_i + (1 - \bbP(w_i)) &= (0.8x + 0.2)(0.5x+0.5)(0.2x+0.8)(0.1x+0.9)(0.3x+0.7)\\
					&= (0.4x^2+0.5x+0.1)(0.2x+0.8)(0.1x+0.9)(0.3x+0.7)\\
					&= (0.08x^3+0.42x^2+0.42x+0.08)(0.1x+0.9)(0.3x+0.7)\\
					&= (0.008x^4+0.114x^2+0.42x^2+0.386x+0.072)(0.3x+0.7)\\
					&= 0.0024x^5 + 0.0398x^4 + 0.2058x^3 + 0.4098x^2 + 0.2918x+0.0504
				\end{align*}
				This means that $\bbP(w=5) = 0.00024, \bbP(w=4) = 0.0398, \bbP(w=3)=0.2058, \bbP(w=2) = 0.4098, \bbP(w=1) = 0.2918, \bbP(w=0)=0.0504$. Since we are asked about the $\bbP(w\geq3)$, then we compute
				\begin{align*}
					\bbP(w\geq 3) &= \bbP(w=5) + \bbP(w=4) + \bbP(w=3)\\
					&= 0.0024+0.398+0.2058\\
					&= 0.248
				\end{align*}
			\end{solution}
		\item {\bf Use a polynomial-time algorithm for this purpose. }
		\begin{solution}
			While these computations were done by \textsc{Mathematica}, we briefly discuss how to do these computations in polynomial time. First, observe that multiplying a polynomial of degree $m$ with a polynomial of degree $n$, we have $(m+1)(n+1)$ multiplies and $mn$ additions, which is $\order{2mn + m+n+1} = \order{mn}$. If we do divide and conquer, then we will only have $\lg w$ levels of this algorithm, where $w$ is  the number of binomials we are multiplying (number of wolves for the above problem). This yields a polynomial time algorithm for computing what we want. 
			
			If we want lower, we can recognized that we only have to do 3 multiplications for two binomials, so using divide and conquer algorithms, we can get $\order{n^{\lg3}}$. If we desire faster, we can use the FFT and iFFT (both done in $\order{n\lg n}$ with $n$ evalutations for the interpolation, so doing this process $\lg w$, we end up with an $\order{n\lg^2n}$ algorithm.
		\end{solution} 
		\item {\bf Given your result above. Should you go out there to protect the tourist groups? Or is the probability low
			enough that you can safely take a nap}
			\begin{solution}
				I don't know how to answer this, as it appears to be more of an ethics question then anything else. The probability is low ($\approx 0.25$), but that is still too high for me to risk the loss of human life.
			\end{solution}
		
		
	\end{enumerate}
	
	\section*{Acknowledgements}
	I would like to acknowledge that I checked my final answer and work with fellow CS 584 student, Lance Kennedy.
	
\end{document}
